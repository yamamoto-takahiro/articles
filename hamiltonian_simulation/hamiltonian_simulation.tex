\documentclass[11pt, oneside]{article}   	% use "amsart" instead of "article" for AMSLaTeX format
\usepackage{geometry}                		% See geometry.pdf to learn the layout options. There are lots.
\geometry{letterpaper}                   		% ... or a4paper or a5paper or ... 
%\geometry{landscape}                		% Activate for rotated page geometry
%\usepackage[parfill]{parskip}    		% Activate to begin paragraphs with an empty line rather than an indent
\usepackage{graphicx}				% Use pdf, png, jpg, or eps§ with pdflatex; use eps in DVI mode
								% TeX will automatically convert eps --> pdf in pdflatex		
\usepackage{amssymb}
\usepackage{amsmath}
\usepackage{braket}
%SetFonts

%SetFonts


\title{Hamiltonian Simulation}
\author{Takahiro Yamamoto}
%\date{}							% Activate to display a given date or no date

\begin{document}
\maketitle

\section{Simulating Hamiltonian dynamics}
In quantum mechanics, time evolution of the state $\ket{\psi(t)}$ is governed by the Schrodinger equation,
\begin{equation*} 
i \frac{d}{dt} \ket{\psi(t)} = H(t) \ket{\psi(t)},
\end{equation*}
where $H(t)$  is the Hamiltonian.
Given an initial state $\ket{\psi(0)}$, we can solve this differential equation to determine $\ket{\psi(t)}$ at any time $t$.

For $H$ independent of time, the solution of the Schrodinger equation is
\begin{equation*} 
\ket{\psi(t)} = e^{-i H t} \ket{\psi(0)}.
\end{equation*}

A Hamiltonian acting on $n$ qubits is said to be \textit{efficiently simulated} if for any $t > 0$, $\epsilon > 0$, there is a quantum circuit $U$ consisting of poly($n, t, 1/\epsilon$) gates such that
\begin{equation*} 
||U - e^{-iHt}  || < \epsilon.
\end{equation*}

\begin{equation*} 
\int dt e^{-i \omega t} \bra{\psi(0)} e^{-i H t} \ket{\psi(0)}
\end{equation*}

\section{Trotter-Suzuki}
\section{Quantum walk}
The quantum walk approach gives optimal complexity as a function of the simulation time $t$, while its performance as a function of the required error $\epsilon$ is worse than PF.

\section{Linear combinations of unitaries}
We can achieve complexity poly($\log(1/\epsilon)$) by techniques for implementing linear combinations of unitary operators.

Basic idea of LCU is that given the ability to implement 
\begin{equation*} 
\textrm{SELECT}(W) = \sum_{j= 0} \ket{j} \! \bra{j} \otimes W_j
\end{equation*}
implement $V = \sum \alpha_i W_i$, where each $W_i$ is an easy-to-implement unitary.
For instance, let $V = W_0 + W_1$,
\begin{align*} 
\textrm{SELECT}(W) \ket{+} \ket{\psi} 
&= \frac{1}{\sqrt{2}} (\ket{0} W_0 + \ket{1} W_1) \ket{\psi} \\
&= \frac{1}{2} (\ket{+} (W_0 + W_1) + \ket{-} (W_0 - W_1)) \ket{\psi} \\
&= \frac{1}{2} \ket{+} V \ket{\psi} + \frac{1}{2} \ket{-} (W_0 - W_1)) \ket{\psi},
\end{align*}
which is a probabilistic implementation of $V$.

More generally, suppose we can decompose the given Hamiltonian in the form

\begin{equation*} 
H = \sum^L_{\ell = 1} \alpha_{\ell} H_{\ell},
\end{equation*}
where $\alpha_{\ell}$ are some real positive coefficients and $H_{\ell}$ are both unitary and Hermitian.
This is straightforward if $H$ is $k$-local, since in that case $H$ can be expressed as linear combinations of Pauli operators.

We denote the Taylor series for $e^{-i H t}$ up to time $t$, truncated at order $K$, by
\begin{align*} 
\tilde{U}(t) 
&= \sum^K_{k = 0} \frac{(-iHt)^k}{k!} \\
&= I + (-it) \sum^L_{\ell = 1} \alpha_{\ell} H_{\ell} + \cdots + \frac{(-it)^K}{K!} \left( \sum^L_{\ell = 1} \alpha_{\ell} H_{\ell} \right)^K \\
&= \sum^K_{k = 0} \sum^L_{\ell_1 = 1} \cdots \sum^L_{\ell_k = 1} \frac{t^k}{k!} \alpha_{\ell_1} \cdots \alpha_{\ell_k} (-i)^k H_{\ell_1} \cdots H_{\ell_k} \\
&= \sum^{m-1}_{j = 0} \beta_j V_j
\end{align*}

Let $B$ be an operator that prepares the state
\begin{equation*} 
B \ket{0} = \ket{\beta} = \frac{1}{s} \sum^{m-1}_{j = 0} \sqrt{\beta_j} \ket{j},
\end{equation*}
where 
\begin{align*} 
s 
&= \sum^{m-1}_{j = 0} \beta_j \\
&= \sum^K_{k = 0} \sum^L_{\ell_1 = 1} \cdots \sum^L_{\ell_k = 1} \frac{t^k}{k!} \alpha_{\ell_1} \cdots \alpha_{\ell_k} \\
&= \sum^K_{k = 0}  \frac{t^k}{k!} \left( \sum^L_{\ell = 1} \alpha_{\ell} \right)^k.
\end{align*}

Let 
\begin{equation*} 
\textrm{SELECT}(V) = \sum^{m-1}_{j = 0} \ket{j} \! \bra{j} \otimes V_j
\end{equation*}
and 
\begin{equation*} 
W = (B^{\dagger} \otimes I) \textrm{SELECT}(V) (B \otimes I)
\end{equation*}

Then we have 
\begin{align*} 
(\bra{0} \otimes I) W (\ket{0} \otimes \ket{\psi}) 
&= (\bra{0} \otimes I) B^{\dagger} \textrm{SELECT}(V) B (\ket{0} \otimes \ket{\psi}) \\
&= \frac{1}{s} \left( \sum^{m-1}_i \sqrt{\beta_i} \bra{i} \otimes I \right) \textrm{SELECT}(V) \left( \sum^{m-1}_j \sqrt{\beta_k} \ket{k} \otimes \ket{\psi} \right) \\
&= \frac{1}{s} \left( \sum^{m-1}_i \sqrt{\beta_i} \bra{i} \otimes I \right) \left( \sum^{m-1}_{j = 0} \ket{j} \! \bra{j} \otimes V_j \right) \left( \sum^{m-1}_j \sqrt{\beta_k} \ket{k} \otimes \ket{\psi} \right) \\
&= \frac{1}{s} \sum^{m-1}_{j = 0} \beta_j V_j \ket{\psi} \\
&= \frac{1}{s} \tilde{U}(t) \ket{\psi}
\end{align*}
If we postselect the state $W (\ket{0} \otimes \ket{\psi})$ on having its first register in the state $\ket{\psi}$, we obtain the desired result, with the success probability of approximately $1/s^2$.
$W$ is called \textit{probabilistic implementation} of $U$ with probability $1/s$, or $W$ \textit{block-encodes} the operator $U/s$.

The action of $W$ on the full space is
\begin{equation*} 
W (\ket{0} \otimes \ket{\psi}) = \frac{1}{s} \ket{0} \otimes \tilde{U}(t) \ket{\psi} + \sqrt{1 - \frac{1}{s^2}} \ket{\Phi}
\end{equation*}
where subspace of $\ket{\Phi}$ is orthogonal to $\ket{0}$, or
\begin{equation*}
(\ket{0} \! \bra{0} \otimes I) \ket{\Phi} = 0
\end{equation*}

To boost the chance of success, we would like to apply amplitude amplification to $W$. Note however that $\ket{\psi}$, about which we would like to reflect, is unknown.
Alternatively we can apply the reflection about the subspace $\ket{0}$
\begin{equation*} 
R = (I - 2 \ket{0} \! \bra{0}) \otimes I
\end{equation*}
Let the projection $P = \ket{0} \! \bra{0}$, we have
\begin{align*} 
W R W^{\dagger} R W 
&= W ((I - 2 P) \otimes I) W^{\dagger} ((I - 2 P) \otimes I) W \\
&= W W^{\dagger} W - 2 W P W^{\dagger} W - 2 W W^{\dagger} P W + 4 W P W^{\dagger} P W,
\end{align*}
hence 
\begin{align*} 
(\bra{0} \otimes I) W R W^{\dagger} R W (\ket{0} \otimes I) 
&= (\bra{0} \otimes I) (-3W + 4 W P W^{\dagger} P W) (\ket{0} \otimes I) 
\end{align*}

TODO: check
\begin{enumerate}
\item
Is $W^{\dagger} = W^{-1}$?
\item
$\bra{0} B^{\dagger} B \ket{0} = 1$, then what is $B^{\dagger} \ket{0}$ ?
\end{enumerate}

Therefore
\begin{equation*} 
(\bra{0} \otimes I) W R W^{\dagger} R W (\ket{0} \otimes \ket{\psi}) 
= - \frac{3}{s} \tilde{U}(t) \ket{\psi} + \frac{4}{s^3} \tilde{U}(t) \tilde{U}^{\dagger}(t) \tilde{U}(t) \ket{\psi}, 
\end{equation*}
which is close to $-(3/s - 4/s^3) \tilde{U}(t)$ since $\tilde{U}(t)$ is close to unitary.
For the purpose of Hamiltonian simulation, we can choose the parameters such that a single segment of the evolution has the value of $s$, and we repeat the process, called \textit{oblivious amplitude amplification}. 
More generally, the operation $W R W^{\dagger} R W$ is applied many times to boost the amplitude for success to a value close to unity. 
LCU can be implemented with complexity $O(1/s)$.
It is important to note that $U$ is (closed to) unitary for OAA to work.


\section{Quantum signal processing}
Suppose we can decompose the given Hamiltonian in the form

\begin{equation*} 
H = \sum^L_{\ell = 1} \alpha_{\ell} H_{\ell},
\end{equation*}
where $\alpha_{\ell}$ are some real positive coefficients and $H_{\ell}$ are both unitary and Hermitian.

Let 
\begin{equation*} 
\textrm{SELECT}(H) = \sum^{L}_{\ell = 1} \ket{\ell} \! \bra{\ell} \otimes H_{\ell}
\end{equation*}
and 
\begin{equation*} 
\textrm{PREPARE} \ket{0} = \frac{1}{\sqrt{\alpha}} \sum^{L}_{\ell = 1} \sqrt{\alpha_{\ell}} \ket{\ell} = \ket{G},
\end{equation*}
where $\alpha = \sum^L_{\ell = 1} \alpha_{\ell}$.
Then we have
\begin{align*} 
(\bra{G} \otimes I) \textrm{SELECT}(H) (\ket{G} \otimes I) 
&= \left( \frac{1}{\sqrt{\alpha}} \sum^{L}_{j = 1} \sqrt{\alpha_{j}} \bra{j} \otimes I \right) \sum^{L}_{\ell = 1} \ket{\ell} \! \bra{\ell} \otimes H_{\ell} \left( \frac{1}{\sqrt{\alpha}} \sum^{L}_{k = 1} \sqrt{\alpha_{k}} \ket{k} \otimes I \right) \\
&=  \frac{1}{\alpha} \sum^{L}_{\ell = 1} \alpha_{\ell} H_{\ell} \\
&=  \frac{1}{\alpha} H
\end{align*}

Let the spectral decompositions of $H/\alpha$ is
\begin{equation*} 
\frac{H}{\alpha} = \sum_{\lambda} \lambda \ket{\lambda} \! \bra{\lambda},
\end{equation*}
where the sum runs over all eigenvalues of $H/\alpha$ and $|\lambda| \leq 1$.

The concept of \textit{qubitization} relates the spectral decompositions of $H/\alpha$ and 
\begin{equation*} 
\mathcal{W} = \left( (2 \ket{G} \! \bra{G} - I) \otimes I \right) \textrm{SELECT}(H).
\end{equation*}
\begin{equation*} 
\mathcal{R} = \left( (2 \ket{G} \! \bra{G} - I) \otimes I \right)
\end{equation*}

Theorem 2 of [Low and Chuang 2016] asserts that for each eigenvalue $\lambda \in (-1, 1)$, $\mathcal{W}$ has two corresponding eigenvalues (TODO: proof)

\begin{equation*} 
\lambda_{\pm} = \mp \sqrt{1 - \lambda^2} - i \lambda = \mp e^{\pm i \arcsin(\lambda)},
\end{equation*}
with eigenvectors $\ket{\lambda_{\pm}} = (\ket{G_{\lambda}} \pm i \ket{G^{\perp}_{\lambda}})/\sqrt{2}$, where
\begin{align*} 
\ket{G_{\lambda}} &= \ket{G} \otimes \ket{\lambda} \\
\ket{G^{\perp}_{\lambda}} &= \frac{\lambda \ket{G_{\lambda}} - \textrm{SELECT}(H) \ket{G_{\lambda}}}{\sqrt{1 - \lambda^2}}
\end{align*}
(TODO: proof)
\begin{equation*} 
\mathcal{W} = e^{i \arccos(\lambda) Y}
\end{equation*}

To perform Hamiltonian simulation by qubitization, we implement a function of $\theta$ that converts the eigenvalues $\lambda_{\pm}$ of $-iQ$ to the desired phase 
$e^{-i \lambda t}$, where %namely the Jacobi-Anger expantion
% \begin{equation*} 
% e^{-i \sin(\theta) t} = \sum^{\infty}_{k=-\infty} J_k(t) e^{i k \theta}
% \end{equation*}

\begin{equation*} 
\theta (\lambda_{\pm}) = \mp \arccos(\lambda)
\end{equation*}
We approximate $e^{-i \lambda t}$ with the Jacobi-Anger expansion
\begin{equation*} 
e^{-i \cos(z) t} = \sum^{\infty}_{k=-\infty} i^k J_k(t) e^{i k z}
\end{equation*}
where $J_k(t)$ are Bessel function of the first kind.
By identifying $\cos(z) = \lambda$, we obtain
\begin{align*} 
e^{-i \lambda t} 
&= \sum^{\infty}_{k=-\infty} i^k J_k(t) e^{i k \arccos(\lambda)} \\
&= J_0(t) + 2 \sum^{\infty}_{k=1} (-1)^{k} J_{2k}(t) T_{2k}(\lambda) + 2i \sum^{\infty}_{k=1} (-1)^{k-1} J_{2k-1}(t) T_{2k-1}(\lambda) \\
&= \mathcal{A}(\lambda) + i \mathcal{C}(\lambda)
\end{align*}
where $T_k (x) = \cos(k \arccos(x))$ is the Chebyshev polynomials.
\begin{equation*} 
T_k (\lambda) = \cos(k \arccos(\lambda)) = \cos \left( k \left( \frac{\pi}{2} + \theta(\lambda_{\pm}) \right) \right)
\end{equation*}
hence 
\begin{equation*}
e^{-i \lambda t} = \mathcal{A} \left( \frac{\pi}{2} + \theta(\lambda_{\pm}) \right) + i \mathcal{C} \left( \frac{\pi}{2} + \theta(\lambda_{\pm}) \right)
\end{equation*}

The QSP algorithm applies a sequence of \textit{phased iterates}. 
We introduce an additional ancilla qubit and define the operator.
Given any unitary $V$ with eigenstates $V \ket{\lambda} = e^{i \theta_{\lambda}} \ket{\lambda}$ and 
\begin{equation*} 
V_0 = \ket{+} \! \bra{+} \otimes I +  \ket{-} \! \bra{-} \otimes V
\end{equation*}
controlled by the single-qubit ancilla register where $X \ket{\pm} = \pm \ket{\pm}$.
\begin{equation*} 
V_{\varphi} = (e^{-i \varphi Z/2} \otimes I) V_0 (e^{i \varphi Z/2} \otimes I) 
\end{equation*}

% \begin{equation*}
% V_{\phi} = (e^{-i \phi Z/2} \otimes I) (\ket{+} \! \bra{+} \otimes I +  \ket{-} \! \bra{-} \otimes (-iQ)) (e^{-i \phi Z/2} \otimes I)
% \end{equation*}

To simulate evolution of an initial state $\ket{\psi}$, the QSP algorithm applies $V$ to the state $\ket{+} \otimes \ket{G} \otimes \ket{\psi}$
\begin{align*} 
e^{-i \phi Z/2} \ket{+}
&= e^{-i \phi/2} (\ket{0} + e^{-i \phi/2} \ket{1}) \\
&= e^{-i \phi/2} /2 (1+e^{-i \phi/2}) \ket{+} + (1-e^{-i \phi/2}) \ket{-} \\
&= e^{-i \phi/2}  (\cos(\phi/2) \ket{+} + i \sin(\phi/2) \ket{-})
\end{align*}

\begin{align*} 
V_{\phi} (\ket{+} \otimes \ket{G} \otimes \ket{\lambda})
&= (e^{-i \phi Z/2} \otimes I) (\ket{+} \! \bra{+} \otimes I +  \ket{-} \! \bra{-} \otimes (-iQ)) (e^{-i \phi Z/2} \otimes I) (\ket{+} \otimes \ket{G} \otimes \ket{\lambda}) \\
&= e^{i \phi} (e^{-i \phi Z/2} \otimes I) (\ket{+} \! \bra{+} \otimes I +  \ket{-} \! \bra{-} \otimes (-iQ)) (\cos(\phi/2) \ket{+} + i \sin(\phi/2) \ket{-}) \otimes \ket{G} \otimes \ket{\lambda}) \\
&=  (e^{-i \phi Z/2} \otimes I) (\cos(\phi/2) \ket{+} + ie^{-i \theta_{\lambda}} \sin(\phi/2) \ket{-}) \otimes \ket{G} \otimes \ket{\lambda}) \\
&= (\cos(\theta_{\lambda}) \ket{+} + \sin(\theta_{\lambda}) \ket{-}) \otimes \ket{G} \otimes \ket{\lambda})
\end{align*}
and post-selects the ancilla register of the output on the $\ket{+} \otimes \ket{G}$.

Consider the sequence,
\begin{equation*}
V_{\bar{\varphi}} = V^{\dagger}_{\varphi_Q+\pi} V_{\varphi_{Q-1}} \cdots V^{\dagger}_{\varphi_2+\pi} V_{\varphi_{1}}
\end{equation*}
(TODO: add cancellation of phase)
For each eigenstate $\ket{\lambda}$, we obtain a product of single qubit operators 
$R_{\varphi_Q} (\theta_{\lambda}) \cdots R_{\varphi_1} (\theta_{\lambda})$
acting only on the ancilla $\ket{+}$.
The choice of $\{\varphi_1, \cdots, \varphi_Q \}$ determines the effective single-qubit ancilla operator
\begin{equation*}
V_{\bar{\varphi}} = \oplus_{\lambda} \left( \mathcal{A}(\theta_{\lambda}) I + i \mathcal{B}(\theta_{\lambda}) Z + i \mathcal{C}(\theta_{\lambda}) X + i \mathcal{D}(\theta_{\lambda}) Y \right) \otimes \ket{\lambda} \! \bra{\lambda}
\end{equation*}

\begin{equation*}
(\bra{G} \otimes \bra{+}) V_{\bar{\varphi}} (\ket{+} \otimes \ket{G}) = \oplus_{\lambda, \pm} 
\frac{1}{2} \left( \mathcal{A} \left( \frac{\pi}{2} + \theta_{\lambda_{\pm}} \right) I + i \mathcal{C} \left(\frac{\pi}{2} + \theta_{\lambda_{\pm}} \right) X \right) \otimes \ket{\lambda} \! \bra{\lambda}
\end{equation*}

\begin{align*} 
V_{\phi} 
&= (e^{-i \phi Z/2} \otimes I) (\ket{+} \! \bra{+} \otimes I +  \ket{-} \! \bra{-} \otimes (-iQ)) (e^{-i \phi Z/2} \otimes I) \\
&= \sum_{\nu} e^{i \theta_{\nu}/2} R_{\phi} (\theta_{\nu}) \otimes \ket{\nu} \! \bra{\nu}
\end{align*}
where 
\begin{equation*}
R_{\phi} (\theta) = e^{-i \theta (\cos\phi X + \sin\phi Y)/2} 
\end{equation*}

\section{Equations for slides}
\begin{align*} 
H \ket{n} &= E_n  \ket{n} \\
e^{-i H t} \ket{n} &= e^{-i E_n t} \ket{n} = e^{-i \phi_n} \ket{n}
\end{align*}

\begin{equation*}
\frac{\tilde{\phi}_n}{2 \pi} = \frac{j_1}{2} + \frac{j_2}{2^2} + \cdots + \frac{j_t}{2^t} = 0.j_1 j_2 \dots j_t
\end{equation*}

\begin{align*} 
\ket{\psi} \ket{0}^{\otimes t} \xrightarrow{\textrm{QPE}} \ket{\psi} \ket{\tilde{\phi}_n} = \ket{\psi} \ket{j_1 j_2 \dots j_t}
\end{align*}


\begin{align*} 
H \ket{\psi} & = \sum_n c_n H \ket{n} = \sum_n c_n E_n  \ket{n} \\
e^{-i H t} \ket{\psi} &= \sum_n c_n e^{-i E_n t} \ket{n} = \sum_n c_n e^{-i \phi_n} \ket{n}
\end{align*}

\begin{align*} 
\ket{+} \ket{n} 
& \rightarrow \frac{1}{\sqrt{2}} (\ket{0} + e^{-i \phi_n} \ket{1}) \ket{n} \\
& \rightarrow \frac{1}{2} \left[ (1+e^{-i \phi_n}) \ket{0} + (1-e^{-i \phi_n}) \ket{1} \right] \ket{n} 
\end{align*}

\begin{align*} 
\textrm{Prob}(0) &= 1 + \cos(\phi_n) \\
\textrm{Prob}(1) &= 1 - \cos(\phi_n)
\end{align*}

\begin{align*} 
\ket{\psi} &= \sum_n c_n \ket{n} \\
\ket{+} \ket{\psi} 
& \rightarrow \frac{1}{\sqrt{2}} \sum_n c_n (\ket{0} + e^{-i \phi_n} \ket{1}) \ket{n} \\
& \rightarrow \frac{1}{2} \sum_n c_n \left[ (1+e^{-i \phi_n}) \ket{0} + (1-e^{-i \phi_n}) \ket{1} \right] \ket{n} 
\end{align*}


\begin{equation*}
U(t) = e^{-iHt}
\end{equation*}
\begin{equation*}
e^{-iHt} = \left( \prod^L_{\ell=1} e^{-i \alpha_{\ell} H_{\ell} t/\rho} \right)^{\rho}
\end{equation*}

\begin{equation*}
e^{A} e^{B} = e^{A + B + \frac{1}{2} [A, B] + \frac{1}{2} [A, [A, B]] + \frac{1}{2} [B, [B, A]] + \cdots}
\end{equation*}

\begin{align*}
e^{-i (H_0 + H_1) t} &= e^{-i H_0 t} e^{-i H_1 t} + \mathcal{O}(t^2) \\
e^{-i (H_0 + H_1) t} &= e^{-i H_0 t/2} e^{-i H_1 t} e^{-i H_0 t/2} + \mathcal{O}(t^3)
\end{align*}

\begin{equation*}
W = e^{i \theta_{\lambda}} \ket{\lambda} \! \bra{\lambda} 
\end{equation*}

\begin{equation*}
f(W) = \sum_{\lambda} f(\lambda) \ket{\lambda} \! \bra{\lambda} = \sum_{\lambda} e^{-i \lambda t} \ket{\lambda} \! \bra{\lambda}
\end{equation*}

\begin{equation*}
e^{-i \lambda t} = J_0(t) + 2 \sum^{\infty}_{k=1} (-1)^{k} J_{2k}(t) T_{2k}(\cos \theta_{\lambda}) + 2i \sum^{\infty}_{k=1} (-1)^{k-1} J_{2k-1}(t) T_{2k-1}(\cos \theta_{\lambda}) 
\end{equation*}

\begin{equation*}
\mathcal{A}(\theta_{\lambda}) I + i \mathcal{B}(\theta_{\lambda}) Z + i \mathcal{C}(\theta_{\lambda}) X + i \mathcal{D}(\theta_{\lambda}) Y
\end{equation*}

\begin{align*} 
\mathcal{A}(\lambda)
&\approx J_0(t) + 2 \sum^{Q}_{k=1} (-1)^{k} J_{2k}(t) T_{2k}(\cos \theta_{\lambda}) \\
\mathcal{C}(\lambda)
&\approx 2 \sum^{Q}_{k=1} (-1)^{k-1} J_{2k-1}(t) T_{2k-1}(\cos \theta_{\lambda})
\end{align*}

\begin{equation*}
\mathcal{W} = e^{-i \arccos (H/\alpha) Y}
\end{equation*}

\begin{equation*}
f (e^{-i \arccos (H/\alpha)}) = e^{-i H t}
\end{equation*}

\begin{equation*}
\alpha = \sum^L_{\ell=1} | \alpha_{\ell} |
\end{equation*}


% \begin{align*} 
% \end{align*}


% Symbolic integration with respect to the Haar measure on the unitary group
% Zbigniew Puchała, Jarosław Adam Miszczak
% https://arxiv.org/abs/1109.4244
\end{document}  