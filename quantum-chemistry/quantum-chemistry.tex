\documentclass[11pt, oneside]{article}   	% use "amsart" instead of "article" for AMSLaTeX format
\usepackage{geometry}                		% See geometry.pdf to learn the layout options. There are lots.
\geometry{letterpaper}                   		% ... or a4paper or a5paper or ... 
%\geometry{landscape}                		% Activate for rotated page geometry
%\usepackage[parfill]{parskip}    		% Activate to begin paragraphs with an empty line rather than an indent
\usepackage{graphicx}				% Use pdf, png, jpg, or eps§ with pdflatex; use eps in DVI mode
								% TeX will automatically convert eps --> pdf in pdflatex		
\usepackage{amssymb}
\usepackage{amsmath}
\usepackage{braket}
\usepackage{chemformula}
%SetFonts

%SetFonts


\title{Note on Quantum Chemistry}
\author{Takahiro Yamamoto}
%\date{}							% Activate to display a given date or no date
\begin{document}
\maketitle
\section{Introduction}
%\subsection{}
\section{Molecular orbital models}
\subsection{PEA}
\subsection{Hamiltonian reduction methods}
To address molecular problems on our quantum processor, we rely on a compact encoding of the second- quantized fermionic Hamiltonians on to qubits. 

\subsection{Hydrogen molecule}
The \ch{H2} molecular Hamiltonian has 4 spin-orbitals, representing the spin-degenerate 1$s$ orbitals of the two Hydrogen atoms. 
By useing a binary tree encoding [12], the map to a 4 qubit system can be reduced to 2 qubit system due to the spin-parities of the system [9]. 
% [12] Bravyi, S. & Kitaev, A. Fermionic quantum computation. Ann. Phys. 298, 210–226 (2002).
% [9] Bravyi, S., Gambetta, J. M., Mezzacapo, A. & Temme, K. Tapering off qubits to simulate fermionic hamiltonians. arXiv preprint arXiv:1701.08213 (2017).

\subsection{Beryllium hydroride molecule}
The \ch{BeH2} Hamiltonian is defined upon the 1$s$, 2$s$, 2$p_x$ orbitals associated to \ch{Be}, assuming zero filling for the 2$p_y$ and 2$p_z$ orbitals since they do not interact strongly with the subset of orbitals considered, and 1$s$ orbital associated to each \ch{H} atom, for a total of 10 spin orbitals. 
We then assume perfect filling of the two innermost 1$s$ spin-orbitals of \ch{Be}, after dressing them via the diagonalization of the non-interacting part of the fermionic Hamiltonian. 
We map the 8 spin-orbital Hamiltonian of \ch{BeH2} spin-orbital Hamiltonian using the parity mapping, and remove, as in the case of \ch{H2}, two qubits associated to the spin-parity symmetries, 
reducing this to a 6 qubit problem that encodes 8 spin-orbitals. 

\subsection{Lithium hydroride molecule}
A similar approach is also used to map \ch{LiH} onto 4 qubits. 

The Hamiltonians for \ch{H2}, \ch{LiH} and \ch{BeH2} at their equilibrium distance are explicitly given in the Supplementary Information (TABLE S2) of [arXiv: 1704.05018] and the derivation of them are given at Appendix III.

\subsection{BK-tree}

\subsection{VQE}
\subsubsection{Ground state}
\subsubsection{Excited state}
\subsubsection{Ansatz}
\subsubsection{UCC}
Calculation  check on Appendix B of [arXiv: 1805.04340]
\begin{align} 
T_1 &= \sum_{ij} \theta_{ij} (a^{\dagger}_i a_j - a^{\dagger}_j a_i) \\
T_2 &= \sum_{ijkl} \theta_{ijkl} (a^{\dagger}_i a^{\dagger}_j a_k a_l - a^{\dagger}_l a^{\dagger}_k a_j a_i) 
\end{align}
After the Jordan-Wigner transformation for $N$ qubits, which is given by:
\begin{align} 
a_j &= 1^{\otimes j} \otimes \frac{1}{2} (X + i Y) \otimes Z^{\otimes N - j -1} \\
a^{\dagger}_j &= 1^{\otimes j} \otimes \frac{1}{2} (X - i Y) \otimes Z^{\otimes N - j -1} 
\end{align}
then for $i > j$
\begin{align} 
a^{\dagger}_i &= 1^{\otimes j} \otimes 1^{\otimes i-j} \otimes  \frac{1}{2} (X + i Y) \otimes Z^{\otimes N - i -1} \\
a_j &= 1^{\otimes j} \otimes \frac{1}{2} (X - i Y) \otimes Z^{\otimes i - j} \otimes Z^{\otimes N - i -1}
\end{align}
\begin{align} 
a^{\dagger}_i a_j  
&= 1^{\otimes j} \otimes \frac{1}{2} (X - i Y) \otimes Z^{\otimes i-j-1} \otimes \frac{1}{2} (X + i Y) Z \otimes 1^{\otimes N - i -1} \\
&= 1^{\otimes j} \otimes \frac{1}{2} (X - i Y) \otimes Z^{\otimes i-j-1} \otimes \frac{1}{2} (-Y + i X) \otimes 1^{\otimes N - i -1} \\
a^{\dagger}_j a_i 
&= 1^{\otimes j} \otimes \frac{1}{2} (X + i Y) \otimes Z^{\otimes i - j-1}  \otimes \frac{1}{2} Z (X - i Y) \otimes 1^{\otimes N - i -1} \\
&= 1^{\otimes j} \otimes \frac{1}{2} (X + i Y) \otimes Z^{\otimes i-j-1} \otimes \frac{1}{2} (Y + i X) \otimes 1^{\otimes N - i -1} 
\end{align}
Then
\begin{align}
a^{\dagger}_i a_j - a^{\dagger}_j a_i 
&= 1^{\otimes j} \otimes  \frac{1}{2} 
\left[ X \otimes Z^{\otimes i-j-1} \otimes Y - Y \otimes Z^{\otimes i-j-1} \otimes X \right] 
\otimes 1^{\otimes N - i -1} \\
& = \frac{i}{2} \bigotimes^{i-1}_{a=j+1} Z_a \left[ Y_j X_i - X_j Y_i  \right] 
\end{align}
Since 
\begin{equation}
[Y_j X_i, X_j Y_i] = Y_j X_i X_j Y_i - X_j Y_i Y_j X_i = - Z_j Z_i + Z_j Z_i  = 0,
\end{equation}
Therefore
\begin{align}
\prod_{ij} \exp \left[ \theta_{ij} (a^{\dagger}_i a_j - a^{\dagger}_j a_i) \right] 
&= \prod_{i > j} \exp \left[ \frac{i}{2} \theta_{ij} \bigotimes^{i-1}_{a=j+1} Z_a  (Y_j X_i - X_j Y_i) \right] \\
&= \prod_{i>j} 
\exp \left[ \frac{i}{2} \theta_{ij} \bigotimes^{i-1}_{a=j+1} Z_a  (Y_j X_i) \right] 
\exp \left[ - \frac{i}{2} \theta_{ij} \bigotimes^{i-1}_{a=j+1} Z_a  (X_j Y_i) \right]
\end{align}

Suppose we choose a set of $(a^{\dagger}_i a_j - a^{\dagger}_j a_i)$ so each of them conserves $s_z$, and the total number of electrons.
\begin{align}
\exp \left[ \frac{i}{2} \theta \bigotimes^{i-1}_{a=j+1} Z_a  (Y_j X_i) \right] 
&= \cos \left( \frac{\theta}{2} \right) 1 + i \sin \left( \frac{\theta}{2} \right) \bigotimes^{i-1}_{a=j+1} Z_a  (Y_j X_i) \\
\exp \left[ - \frac{i}{2} \theta \bigotimes^{i-1}_{a=j+1} Z_a  (X_j Y_i) \right]
&= \cos \left( \frac{\theta}{2} \right) 1 - i \sin \left( \frac{\theta}{2} \right) \bigotimes^{i-1}_{a=j+1} Z_a  (X_j Y_i)
\end{align}

\begin{align}
\label{eq:uccs}
\exp \left[ \frac{i}{2} \theta \bigotimes^{i-1}_{a=j+1} Z_a  (Y_j X_i - X_j Y_i) \right] 
&= \cos^2 \left( \frac{\theta}{2} \right) 1 +  \sin^2 \left( \frac{\theta}{2} \right) (Z_j Z_i) \\
&+ i \cos  \left( \frac{\theta}{2} \right) \sin \left( \frac{\theta}{2} \right) \bigotimes^{i-1}_{a=j+1} Z_a  (Y_j X_i - X_j Y_i) 
\end{align}
Now since $i$ is the index of a virtual orbital and $j$ the index of an occupied orbital, $(Z_j \otimes Z_i)$ on HF state yiels $-1$.
And also,
\begin{align}
\bigotimes^{i-1}_{a=j+1} Z_a Y_j X_i \ket{00\cdots01\cdots11} 
&= \bigotimes^{i-1}_{a=j+1} Z_a (i Z_j X_j) X_i \ket{00\cdots01\cdots11} \\
&= i (-)^{n} \ket{0\cdots 1 \cdots01\cdots1 \cdots 1}
\end{align}
and 
\begin{align}
\bigotimes^{i-1}_{a=j+1} Z_a X_j Y_i \ket{00\cdots01\cdots11} 
&= \bigotimes^{i-1}_{a=j+1} Z_a X_j (i Z_i X_i) \ket{00\cdots01\cdots11} \\
&= - i (-)^{n} \ket{0\cdots 1 \cdots01\cdots1 \cdots 1}
\end{align}
Thus we can simplify Eq.~\ref{eq:uccs} to obtain
\begin{align}
\exp \left[ \frac{i}{2} \theta \bigotimes^{i-1}_{a=j+1} Z_a  (Y_j X_i - X_j Y_i) \right] 
&= \cos^2 \left( \frac{\theta}{2} \right) 1 -  \sin^2 \left( \frac{\theta}{2} \right) 1 \\
&+ 2 i \cos  \left( \frac{\theta}{2} \right) \sin \left( \frac{\theta}{2} \right) \bigotimes^{i-1}_{a=j+1} Z_a  (Y_j X_i) \\
&= \cos (\theta) 1 + i \sin (\theta) \bigotimes^{i-1}_{a=j+1} Z_a  (Y_j X_i) 
\end{align}

On the contrary, if we were to choose $(i, j)$ are both the indices of occupied or virtual orbitals, we obtain $(Z_j \otimes Z_i) = 1$ and 
\begin{align}
\bigotimes^{i-1}_{a=j+1} Z_a X_j Y_i \ket{\mathrm{HF}} 
= \bigotimes^{i-1}_{a=j+1} Z_a Y_j X_i \ket{\mathrm{HF}}.
\end{align}

Therefore Eq.~\ref{eq:uccs} becomes
\begin{equation}
\exp \left[ \frac{i}{2} \theta \bigotimes^{i-1}_{a=j+1} Z_a  (Y_j X_i - X_j Y_i) \right] 
= \cos^2 \left( \frac{\theta}{2} \right) 1 +  \sin^2 \left( \frac{\theta}{2} \right) 1
= 1
\end{equation}

For UCCD, from the relation $Y \ket{0} = -i \ket{1}$ and $Y \ket{1} = i \ket{0}$, we obtain
\begin{align}
X_0 X_1 Y_2 X_3 \ket{0011} &= - i \ket{1100} \\
Y_0 X_1 Y_2 Y_3 \ket{0011} &= (-i)^2 i \ket{1100} \\
X_0 Y_1 Y_2 Y_3 \ket{0011} &= (-i)^2 i \ket{1100} \\
X_0 X_1 X_2 Y_3 \ket{0011} &= - i \ket{1100}
\end{align}
and 
\begin{align}
Y_0 X_1 X_2 X_3 \ket{0011} &= i \ket{1100} \\
X_0 Y_1 X_2 X_3 \ket{0011} &= i \ket{1100} \\
Y_0 Y_1 Y_2 X_3 \ket{0011} &= (-i) i^2 \ket{1100} \\
Y_0 Y_1 X_2 Y_3 \ket{0011} &= (-i) i^2 \ket{1100}
\end{align}

In the same token, if we choose $(k, l)$ are both the indices of occupied orbitals and $(i, j)$ the indices of virtual orbitals, 
\begin{align}
\left( \bigotimes^{k-1}_{b=l+1} Z_b \bigotimes^{i-1}_{a=j+1} Z_a \right) X_l X_k Y_j X_i \ket{\mathrm{HF}}
& = \left( \bigotimes^{k-1}_{b=l+1} Z_b \bigotimes^{i-1}_{a=j+1} Z_a \right) Y_l X_k Y_j Y_i \ket{\mathrm{HF}} \\
& = \left( \bigotimes^{k-1}_{b=l+1} Z_b \bigotimes^{i-1}_{a=j+1} Z_a \right) X_l Y_k Y_j Y_i \ket{\mathrm{HF}} \\
& = \left( \bigotimes^{k-1}_{b=l+1} Z_b \bigotimes^{i-1}_{a=j+1} Z_a \right) X_l X_k X_j Y_i \ket{\mathrm{HF}} \\
& = - \left( \bigotimes^{k-1}_{b=l+1} Z_b \bigotimes^{i-1}_{a=j+1} Z_a \right) Y_l X_k X_j X_i \ket{\mathrm{HF}} \\
& = - \left( \bigotimes^{k-1}_{b=l+1} Z_b \bigotimes^{i-1}_{a=j+1} Z_a \right) X_l Y_k X_j X_i \ket{\mathrm{HF}} \\
& = - \left( \bigotimes^{k-1}_{b=l+1} Z_b \bigotimes^{i-1}_{a=j+1} Z_a \right) Y_l Y_k Y_j X_i \ket{\mathrm{HF}} \\
& = - \left( \bigotimes^{k-1}_{b=l+1} Z_b \bigotimes^{i-1}_{a=j+1} Z_a \right) Y_l Y_k X_j Y_i \ket{\mathrm{HF}}
\end{align}
and 

\begin{align}
&\exp \left[ i \frac{\theta}{8} \bigotimes^{k-1}_{b=l+1} Z_b \bigotimes^{i-1}_{a=j+1} Z_a (X_l X_k Y_j X_i + \cdots - Y_l X_k X_j X_i - \cdots) \right] \\
& = \exp \left[ i \frac{\theta}{2} \bigotimes^{k-1}_{b=l+1} Z_b \bigotimes^{i-1}_{a=j+1} Z_a (X_l X_k Y_j X_i - Y_l X_k X_j X_i) \right] \\
&= \cos^2 \left( \frac{\theta}{2} \right) 1 + \sin^2 \left( \frac{\theta}{2} \right) (Z_l Z_j) \\
&+ i \cos \left( \frac{\theta}{2} \right) \sin \left( \frac{\theta}{2} \right) 
\left( \bigotimes^{k-1}_{b=l+1} Z_b \bigotimes^{i-1}_{a=j+1} Z_a \right) (X_l X_k Y_j X_i - Y_l X_k X_j X_i) \\
&= \cos (\theta) 1 + i \sin (\theta) \left( \bigotimes^{k-1}_{b=l+1} Z_b \bigotimes^{i-1}_{a=j+1} Z_a \right) X_l X_k Y_j X_i
\end{align}

\subsubsection{Heuristic Ansatz}

\subsubsection{Circuit building}
\begin{align}
S X S^{\dagger} &= Y \\
H X Y &= Z \\
H Y H &= -Y \\
\mathrm{CNOT}_{12} (1 \otimes Z_2) \mathrm{CNOT}_{12} &= Z_1 \otimes Z_2
\end{align}
Applying the last relation recursively, we obtain
\begin{align}
Z_1 \otimes Z_2 \otimes \cdots \otimes Z_n  
&= \mathrm{CNOT}_{12} (1 \otimes Z_2 \otimes \cdots \otimes Z_n ) \mathrm{CNOT}_{12} \\
&= \mathrm{CNOT}_{12} \cdots \mathrm{CNOT}_{(n-1), n} (1 \otimes \cdots \otimes Z_n ) \mathrm{CNOT}_{(n-1), n} \cdots \mathrm{CNOT}_{12} 
\end{align}
And using this relation and the fact that $\mathrm{CNOT}^2_{ij} = 1$, we see for instance
\begin{align}
\exp \left[ i \frac{\theta}{2} Z_1 \otimes Z_2 \otimes Z_3 \right] 
= \mathrm{CNOT}_{12} \mathrm{CNOT}_{23} \exp \left( i \frac{\theta}{2} Z_3 \right) \mathrm{CNOT}_{23} \mathrm{CNOT}_{12}
\end{align}

\section{non-MO}

\end{document}  