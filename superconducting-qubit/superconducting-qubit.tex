\documentclass[11pt, oneside]{article}   	% use "amsart" instead of "article" for AMSLaTeX format
\usepackage{geometry}                		% See geometry.pdf to learn the layout options. There are lots.
\geometry{letterpaper}                   		% ... or a4paper or a5paper or ... 
%\geometry{landscape}                		% Activate for rotated page geometry
%\usepackage[parfill]{parskip}    		% Activate to begin paragraphs with an empty line rather than an indent
\usepackage{graphicx}				% Use pdf, png, jpg, or eps§ with pdflatex; use eps in DVI mode
								% TeX will automatically convert eps --> pdf in pdflatex		
\usepackage{amssymb}
\usepackage{amsmath}
\usepackage{braket}
%SetFonts

%SetFonts


\title{Superconducting qubit}
\author{Takahiro Yamamoto}
%\date{}							% Activate to display a given date or no date

\begin{document}
\maketitle

\section{Transmon qubit}
\subsection{Linear LC circuit}
In a simple LC resonant circuit both the inductor L and the capacitor C are linear circuit elements. 
Classical Hamiltonian of an electrical LC circuit is

\begin{equation*} 
H = \frac{Q^2}{2 C} + \frac{\Phi^2}{2 L}
\end{equation*}

Defining the reduced flux 
\begin{equation*} 
\phi = 2 \pi \frac{\Phi}{\Phi_0}
\end{equation*}
and the reduced charge 
\begin{equation*} 
n = \frac{Q}{2 e},
\end{equation*}
we can write down the following quantum-mechanical Hamiltonian for the circuit
\begin{equation*} 
H = 4 E_C n^2 + \frac{1}{2} E_L \phi^2,
\end{equation*}
where 
\begin{equation*} 
E_C = \frac{e^2}{2 C} 
\end{equation*}
is the charging energy required to add ``each'' electron of the Cooper-pair to the island and 
\begin{equation*} 
E_L = \frac{\Phi^2_0}{4 \pi^2 L} 
\end{equation*}
is the inductive energy, where 
\begin{equation*} 
\Phi_0 = \frac{h}{2e}
\end{equation*}
is the superconducting magnetic flux quantum. 
Moreover, the quantum operator $n$ is the excess number of Cooper-pairs on the island, 
and $\phi$--the reduced flux--is denoted the ``gauge-invariant phase'' across the inductor. 
(TODO: what is it?)

Note that this Hamiltonian is analogous to that of a mechanical harmonic oscillator, with mass 
$m = C$ 
and resonant frequency 
$\omega = \sqrt{LC}$, 
which expressed in position, $x$, and momentum, $p$, coordinates takes the form 

\begin{equation*} 
H = \frac{p^2}{2 m} + \frac{m \omega^2 x^2}{2}
\end{equation*}


The Hamiltonian in Eq is identical to the one describing a particle in a one-dimensional quadratic potential, a quantum harmonic oscillator (QHO). 
We can treat $\phi$ as the generalized position coordinate, so that the first term is the kinetic energy and the second term is the potential energy. 

We emphasize that the functional form of the potential energy influences the eigensolutions. 
For example, the fact that this term is quadratic ($U_L \propto \phi^2$) in Eq gives rise to the shape of the potential in Fig. 
The solution to this eigenvalue problem gives an infinite series of eigenstates 
$\ket{k}$, ($k = $ 0, 1, 2, $\dots$),
whose corresponding eigenenergies $E_k$ are all equidistantly spaced, i.e., 
$E_{k+1} - E_k = \hbar \omega_r$, where

\begin{equation*} 
\omega_r = \frac{8 E_L E_C}{\hbar} = \frac{1}{\sqrt{LC}}
\end{equation*}

denotes the resonant frequency of the system.

We may represent these results in a more compact form (second quantization) for the quantum harmonic oscillator (QHO) Hamiltonian
\begin{equation*} 
H = \hbar \omega_r \left( a^{\dagger} a + \frac{1}{2} \right), 
\end{equation*}
where $a^{\dagger}$ ($a$) is the creation (annihilation) operator of a single excitation of the resonator. 
The Hamiltonian in Eq. is written as energy. 
It is, however, often preferred to divide by $\hbar$ so that the expression has units of radian frequency, 
since we will later resonantly drive transitions at a particular frequency or reference the rate at which two systems interact with one another. 
Therefore, from here on, $\hbar$ will be omitted.

The original charge number and phase operators can be expressed as 
\begin{equation*} 
n = n_{zpf} i (a - a^{\dagger} )
\end{equation*}
and
\begin{equation*} 
\phi = \phi_{zpf} i (a + a^{\dagger} ),
\end{equation*}
where
\begin{equation*} 
n_{zpf} = \left( \frac{E_L}{32 E_C} \right)^{1/4}
\end{equation*}
and
\begin{equation*} 
\phi_{zpf} = \left( \frac{2 E_C}{E_L} \right)^{1/4}
\end{equation*}
are the ``zero-point fluctuations'' of the charge and phase variables, respectively. 
Quantum mechanically, the quantum states are represented as wavefunctions that are generally distributed over a range of values of $n$ and $\phi$ and, consequently, the wavefunctions have nonzero standard deviations. 
Such wavefunction distributions are referred to as ``quantum fluctuations,'' and they exist, even in the ground state, where they are called zero-point fluctuations.

\subsection{Non-linear LC circuit}
The linear characteristics of the QHO have a natural limitation in its applications for processing quantum information. 
Before the system can be used as a qubit, we need to be able to define a computational subspace consisting of only two energy states 
(usually the two-lowest energy eigenstates) in between which transitions can be driven without also exciting other levels in the system. 
Since many gate operations, such as single-qubit gates, depend on frequency selectivity, 
the equidistant level-spacing of the QHO, illustrated in Fig., poses a practical limitation.

To mitigate the problem of unwanted dynamics involving non-computational states, 
we need to add anharmonicity (or nonlinearity) into our system.
In short, we require the transition frequencies 
$\omega^{0 \to 1}_q$
and 
$\omega^{1 \to 2}_q$
be sufficiently different to be individually addressable. 
(TODO: how different? Give concrete numbers.)

In general, the larger the anharmonicity the better it is. 
In practice, the amount of anharmonicity sets a limit on how short the pulses used to drive the qubit can be. 
(TODO: why it is so?)

To introduce the nonlinearity required to modify the harmonic potential, 
we use the Josephson junction--a nonlinear, dissipationless circuit element that forms the backbone in superconducting circuits. 
By replacing the linear inductor of the QHO with a Josephson junction(TODO: what is it?), playing the role of a nonlinear inductor, 
we can modify the functional form of the potential energy. 
The potential energy of the Josephson junction can be derived from Eq. and the two Josephson relations
\begin{equation*} 
I = I_C \sin(\phi),
\end{equation*}
\begin{equation*} 
V = \frac{\hbar}{2 e} \frac{d \phi}{dt},
\end{equation*}
resulting in a modified Hamiltonian
\begin{equation*} 
H = 4 E_C n^2 - E_J \cos(\phi),
\end{equation*}
where
\begin{equation*} 
E_C = \frac{e^2}{2 C_{\Sigma}},
\end{equation*}
$C_{\Sigma} = C_{s} + C_{J}$ 
is the total capacitance, including both shunt capacitance $Cs$ and the self-capacitance of the junction $CJ$, and 
\begin{equation*} 
E_J = I_c \frac{\Phi}{2 \pi},
\end{equation*}
is the Josephson energy, with $I_c$ being the critical current of the junction.
After introducing the Josephson junction in the circuit, the potential energy no longer takes a manifestly parabolic form (from which the harmonic spectrum originates), but rather features a cosinusoidal form, 
which makes the energy spectrum nondegenerate. 
Therefore, the Josephson junction is the key ingredient that makes the oscillator anharmonic and thus allows us to identify a uniquely addressable quantum two-level system.

Once the nonlinearity has been added, the system dynamics is governed by the dominant energy in Eq. (16), reflected in the 
$E_J/E_C$ ratio. 
Over time, the superconducting qubit community has converged toward circuit designs with 
$E_j >> E_C$. 
In the opposite case when 
$E_J \leq E_C$ , the qubit becomes highly sensitive to charge noise, 
which has proven more challenging to mitigate than flux noise, making it very hard to achieve high coherence. 
Another motivation is that cur- rent technologies allow for more flexibility in engineering the inductive (or potential) part of the Hamiltonian. 
Therefore, working in the $E_J \leq E_C$ limit, makes the system more sensitive to the change in the potential Hamiltonian. 
Therefore, we will focus here on the state-of-the-art qubit modalities that fall in the regime 
$E_J >> E_C$.

To access the $E_J >> E_C$ regime, 
one preferred approach is to make the charging $E_C$ small by shunting the junction with a large capacitor, (TODO: what is shunt capacitor?)
$C_S >> C_J$, effectively making the qubit less sensitive to charge noise--a circuit commonly known as the transmon qubit. 
In this limit, the superconducting phase $\phi$ is a good quantum number, 
i.e., the spread (or quantum fluctuation) of $\phi$ values represented by the quantum wavefunction is small. 
The low-energy eigenstates are therefore, to a good approximation, localized states in the potential well. 
We may gain more insight by expanding the potential term of Eq. (16) into a power series (since $\phi$ is small), that is
\begin{equation*} 
E_J \cos(\phi) = \frac{1}{2} E_J \phi^2 - \frac{1}{24} E_J \phi^4 + \mathcal{O}(\phi^6),
\end{equation*}
The leading quadratic term alone will result in a QHO. 
The second term, however, is quartic which modifies the eigensolution and disrupts the otherwise harmonic energy structure. 
Note that, the negative coefficient of the quartic term indicates that the anharmonicity 
\begin{equation*} 
\alpha = \omega^{1 \to 2}_q - \omega^{0 \to 1}_q,
\end{equation*}
is negative and its limit in magnitude thus cannot be made arbitrarily large. 
For the case of the transmon, 
$\alpha = - E_C$
is usually designed to be 100–300 MHz, as required to maintain a desirable qubit frequency 
\begin{equation*} 
\omega_q = \frac{\sqrt{8 E_J E_C} - E_C}{\hbar} = 3-6 GHz,
\end{equation*}
while keeping an energy ratio sufficiently large ($E_J/E_C \geq 50$) to suppress charge sensitivity. 
Fortunately, the charge sensitivity is exponentially suppressed for an increased $E_J/E_C$, 
while the reduction in anharmonicity only scales as a weak power law, leading to a workable device.

Including terms up to fourth order and using the QHO eigenbases, 
the system Hamiltonian resembles that of a Duffing oscillator
\begin{equation*} 
H = \omega_q a^{\dagger} a + \frac{\alpha}{2} a^{\dagger} a^{\dagger} a a.
\end{equation*}
Since 
$|\alpha| << \omega_q$, 
we can see that the transmon qubit is basically a weakly anharmonic oscillator (AHO). 
If excitation to higher noncomputational states is suppressed over any gate operations, either due to a large enough
$|\alpha|$ 
or due to robust control techniques such as the derivative reduction by adiabatic gate (DRAG) pulse, 
we may effectively treat the AHO as a quantum two-level system, simplifying the Hamiltonian to
\begin{equation*} 
H = \omega_q \frac{\sigma_z}{2}.
\end{equation*}
However, one should always keep in mind that higher levels physically exist.
Their influence on the system dynamics should be taken into account when designing the system and its control processes. 
In fact, there are many cases where the higher levels have proven useful to implement more efficient gate operations.

\section{Gate operation with pulse}
\subsection{Qubit Drive and the Rotating Wave Approximation}
https://qiskit.org/textbook/ch-quantum-hardware/transmon-physics.html#6.-Qubit-Drive-and-the-Rotating-Wave-Approximation-

\subsection{DRAG pulse}

\section{Measurement with pulse}

\section{Noise}
\subsection{Energy (thermal) Relaxation}

\section{Optimizing Microwave Pulses for High-Fidelity Qubit Operations}

\section{Relic}
Given an initial state $\ket{\psi(0)}$, we can solve this differential equation to determine $\ket{\psi(t)}$ at any time $t$.

For $H$ independent of time, the solution of the Schrodinger equation is
\begin{equation*} 
\ket{\psi(t)} = e^{-i H t} \ket{\psi(0)}.
\end{equation*}

A Hamiltonian acting on $n$ qubits is said to be \textit{efficiently simulated} if for any $t > 0$, $\epsilon > 0$, there is a quantum circuit $U$ consisting of poly($n, t, 1/\epsilon$) gates such that
\begin{equation*} 
||U - e^{-iHt}  || < \epsilon.
\end{equation*}

\begin{equation*} 
\int dt e^{-i \omega t} \bra{\psi(0)} e^{-i H t} \ket{\psi(0)}
\end{equation*}

\section{Trotter-Suzuki}
\section{Quantum walk}
The quantum walk approach gives optimal complexity as a function of the simulation time $t$, while its performance as a function of the required error $\epsilon$ is worse than PF.

\section{Linear combinations of unitaries}
We can achieve complexity poly($\log(1/\epsilon)$) by techniques for implementing linear combinations of unitary operators.

Basic idea of LCU is that given the ability to implement 
\begin{equation*} 
\textrm{SELECT}(W) = \sum_{j= 0} \ket{j} \! \bra{j} \otimes W_j
\end{equation*}
implement $V = \sum \alpha_i W_i$, where each $W_i$ is an easy-to-implement unitary.
For instance, let $V = W_0 + W_1$,
\begin{align*} 
\textrm{SELECT}(W) \ket{+} \ket{\psi} 
&= \frac{1}{\sqrt{2}} (\ket{0} W_0 + \ket{1} W_1) \ket{\psi} \\
&= \frac{1}{2} (\ket{+} (W_0 + W_1) + \ket{-} (W_0 - W_1)) \ket{\psi} \\
&= \frac{1}{2} \ket{+} V \ket{\psi} + \frac{1}{2} \ket{-} (W_0 - W_1)) \ket{\psi},
\end{align*}
which is a probabilistic implementation of $V$.

More generally, suppose we can decompose the given Hamiltonian in the form

\begin{equation*} 
H = \sum^L_{\ell = 1} \alpha_{\ell} H_{\ell},
\end{equation*}
where $\alpha_{\ell}$ are some real positive coefficients and $H_{\ell}$ are both unitary and Hermitian.
This is straightforward if $H$ is $k$-local, since in that case $H$ can be expressed as linear combinations of Pauli operators.

We denote the Taylor series for $e^{-i H t}$ up to time $t$, truncated at order $K$, by
\begin{align*} 
\tilde{U}(t) 
&= \sum^K_{k = 0} \frac{(-iHt)^k}{k!} \\
&= I + (-it) \sum^L_{\ell = 1} \alpha_{\ell} H_{\ell} + \cdots + \frac{(-it)^K}{K!} \left( \sum^L_{\ell = 1} \alpha_{\ell} H_{\ell} \right)^K \\
&= \sum^K_{k = 0} \sum^L_{\ell_1 = 1} \cdots \sum^L_{\ell_k = 1} \frac{t^k}{k!} \alpha_{\ell_1} \cdots \alpha_{\ell_k} (-i)^k H_{\ell_1} \cdots H_{\ell_k} \\
&= \sum^{m-1}_{j = 0} \beta_j V_j
\end{align*}

Let $B$ be an operator that prepares the state
\begin{equation*} 
B \ket{0} = \ket{\beta} = \frac{1}{s} \sum^{m-1}_{j = 0} \sqrt{\beta_j} \ket{j},
\end{equation*}
where 
\begin{align*} 
s 
&= \sum^{m-1}_{j = 0} \beta_j \\
&= \sum^K_{k = 0} \sum^L_{\ell_1 = 1} \cdots \sum^L_{\ell_k = 1} \frac{t^k}{k!} \alpha_{\ell_1} \cdots \alpha_{\ell_k} \\
&= \sum^K_{k = 0}  \frac{t^k}{k!} \left( \sum^L_{\ell = 1} \alpha_{\ell} \right)^k.
\end{align*}

Let 
\begin{equation*} 
\textrm{SELECT}(V) = \sum^{m-1}_{j = 0} \ket{j} \! \bra{j} \otimes V_j
\end{equation*}
and 
\begin{equation*} 
W = (B^{\dagger} \otimes I) \textrm{SELECT}(V) (B \otimes I)
\end{equation*}

Then we have 
\begin{align*} 
(\bra{0} \otimes I) W (\ket{0} \otimes \ket{\psi}) 
&= (\bra{0} \otimes I) B^{\dagger} \textrm{SELECT}(V) B (\ket{0} \otimes \ket{\psi}) \\
&= \frac{1}{s} \left( \sum^{m-1}_i \sqrt{\beta_i} \bra{i} \otimes I \right) \textrm{SELECT}(V) \left( \sum^{m-1}_j \sqrt{\beta_k} \ket{k} \otimes \ket{\psi} \right) \\
&= \frac{1}{s} \left( \sum^{m-1}_i \sqrt{\beta_i} \bra{i} \otimes I \right) \left( \sum^{m-1}_{j = 0} \ket{j} \! \bra{j} \otimes V_j \right) \left( \sum^{m-1}_j \sqrt{\beta_k} \ket{k} \otimes \ket{\psi} \right) \\
&= \frac{1}{s} \sum^{m-1}_{j = 0} \beta_j V_j \ket{\psi} \\
&= \frac{1}{s} \tilde{U}(t) \ket{\psi}
\end{align*}
If we postselect the state $W (\ket{0} \otimes \ket{\psi})$ on having its first register in the state $\ket{\psi}$, we obtain the desired result, with the success probability of approximately $1/s^2$.
$W$ is called \textit{probabilistic implementation} of $U$ with probability $1/s$, or $W$ \textit{block-encodes} the operator $U/s$.

The action of $W$ on the full space is
\begin{equation*} 
W (\ket{0} \otimes \ket{\psi}) = \frac{1}{s} \ket{0} \otimes \tilde{U}(t) \ket{\psi} + \sqrt{1 - \frac{1}{s^2}} \ket{\Phi}
\end{equation*}
where subspace of $\ket{\Phi}$ is orthogonal to $\ket{0}$, or
\begin{equation*}
(\ket{0} \! \bra{0} \otimes I) \ket{\Phi} = 0
\end{equation*}

To boost the chance of success, we would like to apply amplitude amplification to $W$. Note however that $\ket{\psi}$, about which we would like to reflect, is unknown.
Alternatively we can apply the reflection about the subspace $\ket{0}$
\begin{equation*} 
R = (I - 2 \ket{0} \! \bra{0}) \otimes I
\end{equation*}
Let the projection $P = \ket{0} \! \bra{0}$, we have
\begin{align*} 
W R W^{\dagger} R W 
&= W ((I - 2 P) \otimes I) W^{\dagger} ((I - 2 P) \otimes I) W \\
&= W W^{\dagger} W - 2 W P W^{\dagger} W - 2 W W^{\dagger} P W + 4 W P W^{\dagger} P W,
\end{align*}
hence 
\begin{align*} 
(\bra{0} \otimes I) W R W^{\dagger} R W (\ket{0} \otimes I) 
&= (\bra{0} \otimes I) (-3W + 4 W P W^{\dagger} P W) (\ket{0} \otimes I) 
\end{align*}

TODO: check
\begin{enumerate}
\item
Is $W^{\dagger} = W^{-1}$?
\item
$\bra{0} B^{\dagger} B \ket{0} = 1$, then what is $B^{\dagger} \ket{0}$ ?
\end{enumerate}

Therefore
\begin{equation*} 
(\bra{0} \otimes I) W R W^{\dagger} R W (\ket{0} \otimes \ket{\psi}) 
= - \frac{3}{s} \tilde{U}(t) \ket{\psi} + \frac{4}{s^3} \tilde{U}(t) \tilde{U}^{\dagger}(t) \tilde{U}(t) \ket{\psi}, 
\end{equation*}
which is close to $-(3/s - 4/s^3) \tilde{U}(t)$ since $\tilde{U}(t)$ is close to unitary.
For the purpose of Hamiltonian simulation, we can choose the parameters such that a single segment of the evolution has the value of $s$, and we repeat the process, called \textit{oblivious amplitude amplification}. 
More generally, the operation $W R W^{\dagger} R W$ is applied many times to boost the amplitude for success to a value close to unity. 
LCU can be implemented with complexity $O(1/s)$.
It is important to note that $U$ is (closed to) unitary for OAA to work.


\section{Quantum signal processing}
Suppose we can decompose the given Hamiltonian in the form

\begin{equation*} 
H = \sum^L_{\ell = 1} \alpha_{\ell} H_{\ell},
\end{equation*}
where $\alpha_{\ell}$ are some real positive coefficients and $H_{\ell}$ are both unitary and Hermitian.

Let 
\begin{equation*} 
\textrm{SELECT}(H) = \sum^{L}_{\ell = 1} \ket{\ell} \! \bra{\ell} \otimes H_{\ell}
\end{equation*}
and 
\begin{equation*} 
\textrm{PREPARE} \ket{0} = \frac{1}{\sqrt{\alpha}} \sum^{L}_{\ell = 1} \sqrt{\alpha_{\ell}} \ket{\ell} = \ket{G},
\end{equation*}
where $\alpha = \sum^L_{\ell = 1} \alpha_{\ell}$.
Then we have
\begin{align*} 
(\bra{G} \otimes I) \textrm{SELECT}(H) (\ket{G} \otimes I) 
&= \left( \frac{1}{\sqrt{\alpha}} \sum^{L}_{j = 1} \sqrt{\alpha_{j}} \bra{j} \otimes I \right) \sum^{L}_{\ell = 1} \ket{\ell} \! \bra{\ell} \otimes H_{\ell} \left( \frac{1}{\sqrt{\alpha}} \sum^{L}_{k = 1} \sqrt{\alpha_{k}} \ket{k} \otimes I \right) \\
&=  \frac{1}{\alpha} \sum^{L}_{\ell = 1} \alpha_{\ell} H_{\ell} \\
&=  \frac{1}{\alpha} H
\end{align*}

Let the spectral decompositions of $H/\alpha$ is
\begin{equation*} 
\frac{H}{\alpha} = \sum_{\lambda} \lambda \ket{\lambda} \! \bra{\lambda},
\end{equation*}
where the sum runs over all eigenvalues of $H/\alpha$ and $|\lambda| \leq 1$.

The concept of \textit{qubitization} relates the spectral decompositions of $H/\alpha$ and 
\begin{equation*} 
\mathcal{W} = \left( (2 \ket{G} \! \bra{G} - I) \otimes I \right) \textrm{SELECT}(H).
\end{equation*}
\begin{equation*} 
\mathcal{R} = \left( (2 \ket{G} \! \bra{G} - I) \otimes I \right)
\end{equation*}

Theorem 2 of [Low and Chuang 2016] asserts that for each eigenvalue $\lambda \in (-1, 1)$, $\mathcal{W}$ has two corresponding eigenvalues (TODO: proof)

\begin{equation*} 
\lambda_{\pm} = \mp \sqrt{1 - \lambda^2} - i \lambda = \mp e^{\pm i \arcsin(\lambda)},
\end{equation*}
with eigenvectors $\ket{\lambda_{\pm}} = (\ket{G_{\lambda}} \pm i \ket{G^{\perp}_{\lambda}})/\sqrt{2}$, where
\begin{align*} 
\ket{G_{\lambda}} &= \ket{G} \otimes \ket{\lambda} \\
\ket{G^{\perp}_{\lambda}} &= \frac{\lambda \ket{G_{\lambda}} - \textrm{SELECT}(H) \ket{G_{\lambda}}}{\sqrt{1 - \lambda^2}}
\end{align*}
(TODO: proof)
\begin{equation*} 
\mathcal{W} = e^{i \arccos(\lambda) Y}
\end{equation*}

To perform Hamiltonian simulation by qubitization, we implement a function of $\theta$ that converts the eigenvalues $\lambda_{\pm}$ of $-iQ$ to the desired phase 
$e^{-i \lambda t}$, where %namely the Jacobi-Anger expantion
% \begin{equation*} 
% e^{-i \sin(\theta) t} = \sum^{\infty}_{k=-\infty} J_k(t) e^{i k \theta}
% \end{equation*}

\begin{equation*} 
\theta (\lambda_{\pm}) = \mp \arccos(\lambda)
\end{equation*}
We approximate $e^{-i \lambda t}$ with the Jacobi-Anger expansion
\begin{equation*} 
e^{-i \cos(z) t} = \sum^{\infty}_{k=-\infty} i^k J_k(t) e^{i k z}
\end{equation*}
where $J_k(t)$ are Bessel function of the first kind.
By identifying $\cos(z) = \lambda$, we obtain
\begin{align*} 
e^{-i \lambda t} 
&= \sum^{\infty}_{k=-\infty} i^k J_k(t) e^{i k \arccos(\lambda)} \\
&= J_0(t) + 2 \sum^{\infty}_{k=1} (-1)^{k} J_{2k}(t) T_{2k}(\lambda) + 2i \sum^{\infty}_{k=1} (-1)^{k-1} J_{2k-1}(t) T_{2k-1}(\lambda) \\
&= \mathcal{A}(\lambda) + i \mathcal{C}(\lambda)
\end{align*}
where $T_k (x) = \cos(k \arccos(x))$ is the Chebyshev polynomials.
\begin{equation*} 
T_k (\lambda) = \cos(k \arccos(\lambda)) = \cos \left( k \left( \frac{\pi}{2} + \theta(\lambda_{\pm}) \right) \right)
\end{equation*}
hence 
\begin{equation*}
e^{-i \lambda t} = \mathcal{A} \left( \frac{\pi}{2} + \theta(\lambda_{\pm}) \right) + i \mathcal{C} \left( \frac{\pi}{2} + \theta(\lambda_{\pm}) \right)
\end{equation*}

The QSP algorithm applies a sequence of \textit{phased iterates}. 
We introduce an additional ancilla qubit and define the operator.
Given any unitary $V$ with eigenstates $V \ket{\lambda} = e^{i \theta_{\lambda}} \ket{\lambda}$ and 
\begin{equation*} 
V_0 = \ket{+} \! \bra{+} \otimes I +  \ket{-} \! \bra{-} \otimes V
\end{equation*}
controlled by the single-qubit ancilla register where $X \ket{\pm} = \pm \ket{\pm}$.
\begin{equation*} 
V_{\varphi} = (e^{-i \varphi Z/2} \otimes I) V_0 (e^{i \varphi Z/2} \otimes I) 
\end{equation*}

% \begin{equation*}
% V_{\phi} = (e^{-i \phi Z/2} \otimes I) (\ket{+} \! \bra{+} \otimes I +  \ket{-} \! \bra{-} \otimes (-iQ)) (e^{-i \phi Z/2} \otimes I)
% \end{equation*}

To simulate evolution of an initial state $\ket{\psi}$, the QSP algorithm applies $V$ to the state $\ket{+} \otimes \ket{G} \otimes \ket{\psi}$
\begin{align*} 
e^{-i \phi Z/2} \ket{+}
&= e^{-i \phi/2} (\ket{0} + e^{-i \phi/2} \ket{1}) \\
&= e^{-i \phi/2} /2 (1+e^{-i \phi/2}) \ket{+} + (1-e^{-i \phi/2}) \ket{-} \\
&= e^{-i \phi/2}  (\cos(\phi/2) \ket{+} + i \sin(\phi/2) \ket{-})
\end{align*}

\begin{align*} 
V_{\phi} (\ket{+} \otimes \ket{G} \otimes \ket{\lambda})
&= (e^{-i \phi Z/2} \otimes I) (\ket{+} \! \bra{+} \otimes I +  \ket{-} \! \bra{-} \otimes (-iQ)) (e^{-i \phi Z/2} \otimes I) (\ket{+} \otimes \ket{G} \otimes \ket{\lambda}) \\
&= e^{i \phi} (e^{-i \phi Z/2} \otimes I) (\ket{+} \! \bra{+} \otimes I +  \ket{-} \! \bra{-} \otimes (-iQ)) (\cos(\phi/2) \ket{+} + i \sin(\phi/2) \ket{-}) \otimes \ket{G} \otimes \ket{\lambda}) \\
&=  (e^{-i \phi Z/2} \otimes I) (\cos(\phi/2) \ket{+} + ie^{-i \theta_{\lambda}} \sin(\phi/2) \ket{-}) \otimes \ket{G} \otimes \ket{\lambda}) \\
&= (\cos(\theta_{\lambda}) \ket{+} + \sin(\theta_{\lambda}) \ket{-}) \otimes \ket{G} \otimes \ket{\lambda})
\end{align*}
and post-selects the ancilla register of the output on the $\ket{+} \otimes \ket{G}$.

Consider the sequence,
\begin{equation*}
V_{\bar{\varphi}} = V^{\dagger}_{\varphi_Q+\pi} V_{\varphi_{Q-1}} \cdots V^{\dagger}_{\varphi_2+\pi} V_{\varphi_{1}}
\end{equation*}
(TODO: add cancellation of phase)
For each eigenstate $\ket{\lambda}$, we obtain a product of single qubit operators 
$R_{\varphi_Q} (\theta_{\lambda}) \cdots R_{\varphi_1} (\theta_{\lambda})$
acting only on the ancilla $\ket{+}$.
The choice of $\{\varphi_1, \cdots, \varphi_Q \}$ determines the effective single-qubit ancilla operator
\begin{equation*}
V_{\bar{\varphi}} = \oplus_{\lambda} \left( \mathcal{A}(\theta_{\lambda}) I + i \mathcal{B}(\theta_{\lambda}) Z + i \mathcal{C}(\theta_{\lambda}) X + i \mathcal{D}(\theta_{\lambda}) Y \right) \otimes \ket{\lambda} \! \bra{\lambda}
\end{equation*}

\begin{equation*}
(\bra{G} \otimes \bra{+}) V_{\bar{\varphi}} (\ket{+} \otimes \ket{G}) = \oplus_{\lambda, \pm} 
\frac{1}{2} \left( \mathcal{A} \left( \frac{\pi}{2} + \theta_{\lambda_{\pm}} \right) I + i \mathcal{C} \left(\frac{\pi}{2} + \theta_{\lambda_{\pm}} \right) X \right) \otimes \ket{\lambda} \! \bra{\lambda}
\end{equation*}

\begin{align*} 
V_{\phi} 
&= (e^{-i \phi Z/2} \otimes I) (\ket{+} \! \bra{+} \otimes I +  \ket{-} \! \bra{-} \otimes (-iQ)) (e^{-i \phi Z/2} \otimes I) \\
&= \sum_{\nu} e^{i \theta_{\nu}/2} R_{\phi} (\theta_{\nu}) \otimes \ket{\nu} \! \bra{\nu}
\end{align*}
where 
\begin{equation*}
R_{\phi} (\theta) = e^{-i \theta (\cos\phi X + \sin\phi Y)/2} 
\end{equation*}

\section{Equations for slides}
\begin{align*} 
H \ket{n} &= E_n  \ket{n} \\
e^{-i H t} \ket{n} &= e^{-i E_n t} \ket{n} = e^{-i \phi_n} \ket{n}
\end{align*}

\begin{equation*}
\frac{\tilde{\phi}_n}{2 \pi} = \frac{j_1}{2} + \frac{j_2}{2^2} + \cdots + \frac{j_t}{2^t} = 0.j_1 j_2 \dots j_t
\end{equation*}

\begin{align*} 
\ket{\psi} \ket{0}^{\otimes t} \xrightarrow{\textrm{QPE}} \ket{\psi} \ket{\tilde{\phi}_n} = \ket{\psi} \ket{j_1 j_2 \dots j_t}
\end{align*}


\begin{align*} 
H \ket{\psi} & = \sum_n c_n H \ket{n} = \sum_n c_n E_n  \ket{n} \\
e^{-i H t} \ket{\psi} &= \sum_n c_n e^{-i E_n t} \ket{n} = \sum_n c_n e^{-i \phi_n} \ket{n}
\end{align*}

\begin{align*} 
\ket{+} \ket{n} 
& \rightarrow \frac{1}{\sqrt{2}} (\ket{0} + e^{-i \phi_n} \ket{1}) \ket{n} \\
& \rightarrow \frac{1}{2} \left[ (1+e^{-i \phi_n}) \ket{0} + (1-e^{-i \phi_n}) \ket{1} \right] \ket{n} 
\end{align*}

\begin{align*} 
\textrm{Prob}(0) &= 1 + \cos(\phi_n) \\
\textrm{Prob}(1) &= 1 - \cos(\phi_n)
\end{align*}

\begin{align*} 
\ket{\psi} &= \sum_n c_n \ket{n} \\
\ket{+} \ket{\psi} 
& \rightarrow \frac{1}{\sqrt{2}} \sum_n c_n (\ket{0} + e^{-i \phi_n} \ket{1}) \ket{n} \\
& \rightarrow \frac{1}{2} \sum_n c_n \left[ (1+e^{-i \phi_n}) \ket{0} + (1-e^{-i \phi_n}) \ket{1} \right] \ket{n} 
\end{align*}


\begin{equation*}
U(t) = e^{-iHt}
\end{equation*}
\begin{equation*}
e^{-iHt} = \left( \prod^L_{\ell=1} e^{-i \alpha_{\ell} H_{\ell} t/\rho} \right)^{\rho}
\end{equation*}

\begin{equation*}
e^{A} e^{B} = e^{A + B + \frac{1}{2} [A, B] + \frac{1}{2} [A, [A, B]] + \frac{1}{2} [B, [B, A]] + \cdots}
\end{equation*}

\begin{align*}
e^{-i (H_0 + H_1) t} &= e^{-i H_0 t} e^{-i H_1 t} + \mathcal{O}(t^2) \\
e^{-i (H_0 + H_1) t} &= e^{-i H_0 t/2} e^{-i H_1 t} e^{-i H_0 t/2} + \mathcal{O}(t^3)
\end{align*}

\begin{equation*}
W = e^{i \theta_{\lambda}} \ket{\lambda} \! \bra{\lambda} 
\end{equation*}

\begin{equation*}
f(W) = \sum_{\lambda} f(\lambda) \ket{\lambda} \! \bra{\lambda} = \sum_{\lambda} e^{-i \lambda t} \ket{\lambda} \! \bra{\lambda}
\end{equation*}

\begin{equation*}
e^{-i \lambda t} = J_0(t) + 2 \sum^{\infty}_{k=1} (-1)^{k} J_{2k}(t) T_{2k}(\cos \theta_{\lambda}) + 2i \sum^{\infty}_{k=1} (-1)^{k-1} J_{2k-1}(t) T_{2k-1}(\cos \theta_{\lambda}) 
\end{equation*}

\begin{equation*}
\mathcal{A}(\theta_{\lambda}) I + i \mathcal{B}(\theta_{\lambda}) Z + i \mathcal{C}(\theta_{\lambda}) X + i \mathcal{D}(\theta_{\lambda}) Y
\end{equation*}

\begin{align*} 
\mathcal{A}(\lambda)
&\approx J_0(t) + 2 \sum^{Q}_{k=1} (-1)^{k} J_{2k}(t) T_{2k}(\cos \theta_{\lambda}) \\
\mathcal{C}(\lambda)
&\approx 2 \sum^{Q}_{k=1} (-1)^{k-1} J_{2k-1}(t) T_{2k-1}(\cos \theta_{\lambda})
\end{align*}

\begin{equation*}
\mathcal{W} = e^{-i \arccos (H/\alpha) Y}
\end{equation*}

\begin{equation*}
f (e^{-i \arccos (H/\alpha)}) = e^{-i H t}
\end{equation*}

\begin{equation*}
\alpha = \sum^L_{\ell=1} | \alpha_{\ell} |
\end{equation*}


% \begin{align*} 
% \end{align*}


% Symbolic integration with respect to the Haar measure on the unitary group
% Zbigniew Puchała, Jarosław Adam Miszczak
% https://arxiv.org/abs/1109.4244
\end{document}  